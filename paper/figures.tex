
%%%%%%%%%%%%%%%%%%%%%%%%%%%%%%%%%%%%%%%%%%%%%%%%%%%%%%%%%%%%%%%%%%%%%%%%%%%%%%%
% FIGURES - FILO-PRIORI V9
%%%%%%%%%%%%%%%%%%%%%%%%%%%%%%%%%%%%%%%%%%%%%%%%%%%%%%%%%%%%%%%%%%%%%%%%%%%%%%%
% Auto-generated: 2025-11-26 13:31:16
%
% Usage: 
%%%%%%%%%%%%%%%%%%%%%%%%%%%%%%%%%%%%%%%%%%%%%%%%%%%%%%%%%%%%%%%%%%%%%%%%%%%%%%%
% FIGURES - FILO-PRIORI V9
%%%%%%%%%%%%%%%%%%%%%%%%%%%%%%%%%%%%%%%%%%%%%%%%%%%%%%%%%%%%%%%%%%%%%%%%%%%%%%%
% Auto-generated: 2025-11-26 13:31:16
%
% Usage: 
%%%%%%%%%%%%%%%%%%%%%%%%%%%%%%%%%%%%%%%%%%%%%%%%%%%%%%%%%%%%%%%%%%%%%%%%%%%%%%%
% FIGURES - FILO-PRIORI V9
%%%%%%%%%%%%%%%%%%%%%%%%%%%%%%%%%%%%%%%%%%%%%%%%%%%%%%%%%%%%%%%%%%%%%%%%%%%%%%%
% Auto-generated: 2025-11-26 13:31:16
%
% Usage: 
%%%%%%%%%%%%%%%%%%%%%%%%%%%%%%%%%%%%%%%%%%%%%%%%%%%%%%%%%%%%%%%%%%%%%%%%%%%%%%%
% FIGURES - FILO-PRIORI V9
%%%%%%%%%%%%%%%%%%%%%%%%%%%%%%%%%%%%%%%%%%%%%%%%%%%%%%%%%%%%%%%%%%%%%%%%%%%%%%%
% Auto-generated: 2025-11-26 13:31:16
%
% Usage: \input{figures.tex} in your main document
%%%%%%%%%%%%%%%%%%%%%%%%%%%%%%%%%%%%%%%%%%%%%%%%%%%%%%%%%%%%%%%%%%%%%%%%%%%%%%%

%------------------------------------------------------------------------------
% Figure 1: APFD Comparison (RQ1)
%------------------------------------------------------------------------------
\begin{figure}[htbp]
\centering
\includegraphics[width=0.9\textwidth]{figures/fig_rq1_apfd_comparison.pdf}
\caption{Box plot comparison of APFD scores across TCP methods on QTA dataset
(N=277 builds). Filo-Priori achieves competitive performance with traditional
baselines while significantly outperforming recency-based approaches
($p < 0.001$, Wilcoxon signed-rank test). Whiskers extend to 1.5 IQR.}
\label{fig:apfd_comparison}
\end{figure}

%------------------------------------------------------------------------------
% Figure 2: Improvement over Random (RQ1)
%------------------------------------------------------------------------------
\begin{figure}[htbp]
\centering
\includegraphics[width=0.85\textwidth]{figures/fig_rq1_improvement.pdf}
\caption{Relative improvement in APFD over random ordering for each TCP method.
Filo-Priori achieves +10.3\% improvement, significantly outperforming Random
($p < 0.001$). Error bars represent 95\% bootstrap confidence intervals.}
\label{fig:improvement}
\end{figure}

%------------------------------------------------------------------------------
% Figure 3: Ablation Study (RQ2)
%------------------------------------------------------------------------------
\begin{figure}[htbp]
\centering
\includegraphics[width=0.9\textwidth]{figures/fig_rq2_ablation.pdf}
\caption{Component contributions to Filo-Priori performance. Graph Attention
(GATv2) provides the largest contribution (+17.0\%), followed by the Structural
Stream (+5.3\%) and Class Weighting (+4.6\%). Red bars indicate statistically
significant contributions ($p < 0.05$, paired t-test).}
\label{fig:ablation}
\end{figure}

%------------------------------------------------------------------------------
% Figure 4: Temporal Cross-Validation (RQ3)
%------------------------------------------------------------------------------
\begin{figure}[htbp]
\centering
\includegraphics[width=0.85\textwidth]{figures/fig_rq3_temporal.pdf}
\caption{Temporal cross-validation results showing consistent APFD performance
across different validation strategies. The narrow range (0.619--0.663) indicates
temporal robustness and minimal concept drift impact. Error bars represent
95\% bootstrap confidence intervals.}
\label{fig:temporal_cv}
\end{figure}

%------------------------------------------------------------------------------
% Figure 5: Hyperparameter Sensitivity (RQ4)
%------------------------------------------------------------------------------
\begin{figure*}[htbp]
\centering
\includegraphics[width=\textwidth]{figures/fig_rq4_sensitivity.pdf}
\caption{Hyperparameter sensitivity analysis. (a) Loss function has the highest
impact on performance ($\Delta$ = 5.9\%). (b) Lower learning rate (3e-5)
outperforms higher rate. (c) Simpler GNN architecture performs better.
(d) 10 selected features optimal. (e) Balanced sampling degrades performance.
(f) Summary of optimal configuration.}
\label{fig:sensitivity}
\end{figure*}

%------------------------------------------------------------------------------
% Figure 6: Qualitative Analysis
%------------------------------------------------------------------------------
\begin{figure}[htbp]
\centering
\includegraphics[width=0.9\textwidth]{figures/fig_qualitative.pdf}
\caption{Qualitative analysis of Filo-Priori performance. (a) APFD score
distribution across 277 builds. (b) First failure detection position distribution.
(c) Failure detection curve showing that running 25\% of tests detects 33.2\%
of failures. (d) Relationship between build size and APFD score.}
\label{fig:qualitative}
\end{figure}

 in your main document
%%%%%%%%%%%%%%%%%%%%%%%%%%%%%%%%%%%%%%%%%%%%%%%%%%%%%%%%%%%%%%%%%%%%%%%%%%%%%%%

%------------------------------------------------------------------------------
% Figure 1: APFD Comparison (RQ1)
%------------------------------------------------------------------------------
\begin{figure}[htbp]
\centering
\includegraphics[width=0.9\textwidth]{figures/fig_rq1_apfd_comparison.pdf}
\caption{Box plot comparison of APFD scores across TCP methods on QTA dataset
(N=277 builds). Filo-Priori achieves competitive performance with traditional
baselines while significantly outperforming recency-based approaches
($p < 0.001$, Wilcoxon signed-rank test). Whiskers extend to 1.5 IQR.}
\label{fig:apfd_comparison}
\end{figure}

%------------------------------------------------------------------------------
% Figure 2: Improvement over Random (RQ1)
%------------------------------------------------------------------------------
\begin{figure}[htbp]
\centering
\includegraphics[width=0.85\textwidth]{figures/fig_rq1_improvement.pdf}
\caption{Relative improvement in APFD over random ordering for each TCP method.
Filo-Priori achieves +10.3\% improvement, significantly outperforming Random
($p < 0.001$). Error bars represent 95\% bootstrap confidence intervals.}
\label{fig:improvement}
\end{figure}

%------------------------------------------------------------------------------
% Figure 3: Ablation Study (RQ2)
%------------------------------------------------------------------------------
\begin{figure}[htbp]
\centering
\includegraphics[width=0.9\textwidth]{figures/fig_rq2_ablation.pdf}
\caption{Component contributions to Filo-Priori performance. Graph Attention
(GATv2) provides the largest contribution (+17.0\%), followed by the Structural
Stream (+5.3\%) and Class Weighting (+4.6\%). Red bars indicate statistically
significant contributions ($p < 0.05$, paired t-test).}
\label{fig:ablation}
\end{figure}

%------------------------------------------------------------------------------
% Figure 4: Temporal Cross-Validation (RQ3)
%------------------------------------------------------------------------------
\begin{figure}[htbp]
\centering
\includegraphics[width=0.85\textwidth]{figures/fig_rq3_temporal.pdf}
\caption{Temporal cross-validation results showing consistent APFD performance
across different validation strategies. The narrow range (0.619--0.663) indicates
temporal robustness and minimal concept drift impact. Error bars represent
95\% bootstrap confidence intervals.}
\label{fig:temporal_cv}
\end{figure}

%------------------------------------------------------------------------------
% Figure 5: Hyperparameter Sensitivity (RQ4)
%------------------------------------------------------------------------------
\begin{figure*}[htbp]
\centering
\includegraphics[width=\textwidth]{figures/fig_rq4_sensitivity.pdf}
\caption{Hyperparameter sensitivity analysis. (a) Loss function has the highest
impact on performance ($\Delta$ = 5.9\%). (b) Lower learning rate (3e-5)
outperforms higher rate. (c) Simpler GNN architecture performs better.
(d) 10 selected features optimal. (e) Balanced sampling degrades performance.
(f) Summary of optimal configuration.}
\label{fig:sensitivity}
\end{figure*}

%------------------------------------------------------------------------------
% Figure 6: Qualitative Analysis
%------------------------------------------------------------------------------
\begin{figure}[htbp]
\centering
\includegraphics[width=0.9\textwidth]{figures/fig_qualitative.pdf}
\caption{Qualitative analysis of Filo-Priori performance. (a) APFD score
distribution across 277 builds. (b) First failure detection position distribution.
(c) Failure detection curve showing that running 25\% of tests detects 33.2\%
of failures. (d) Relationship between build size and APFD score.}
\label{fig:qualitative}
\end{figure}

 in your main document
%%%%%%%%%%%%%%%%%%%%%%%%%%%%%%%%%%%%%%%%%%%%%%%%%%%%%%%%%%%%%%%%%%%%%%%%%%%%%%%

%------------------------------------------------------------------------------
% Figure 1: APFD Comparison (RQ1)
%------------------------------------------------------------------------------
\begin{figure}[htbp]
\centering
\includegraphics[width=0.9\textwidth]{figures/fig_rq1_apfd_comparison.pdf}
\caption{Box plot comparison of APFD scores across TCP methods on QTA dataset
(N=277 builds). Filo-Priori achieves competitive performance with traditional
baselines while significantly outperforming recency-based approaches
($p < 0.001$, Wilcoxon signed-rank test). Whiskers extend to 1.5 IQR.}
\label{fig:apfd_comparison}
\end{figure}

%------------------------------------------------------------------------------
% Figure 2: Improvement over Random (RQ1)
%------------------------------------------------------------------------------
\begin{figure}[htbp]
\centering
\includegraphics[width=0.85\textwidth]{figures/fig_rq1_improvement.pdf}
\caption{Relative improvement in APFD over random ordering for each TCP method.
Filo-Priori achieves +10.3\% improvement, significantly outperforming Random
($p < 0.001$). Error bars represent 95\% bootstrap confidence intervals.}
\label{fig:improvement}
\end{figure}

%------------------------------------------------------------------------------
% Figure 3: Ablation Study (RQ2)
%------------------------------------------------------------------------------
\begin{figure}[htbp]
\centering
\includegraphics[width=0.9\textwidth]{figures/fig_rq2_ablation.pdf}
\caption{Component contributions to Filo-Priori performance. Graph Attention
(GATv2) provides the largest contribution (+17.0\%), followed by the Structural
Stream (+5.3\%) and Class Weighting (+4.6\%). Red bars indicate statistically
significant contributions ($p < 0.05$, paired t-test).}
\label{fig:ablation}
\end{figure}

%------------------------------------------------------------------------------
% Figure 4: Temporal Cross-Validation (RQ3)
%------------------------------------------------------------------------------
\begin{figure}[htbp]
\centering
\includegraphics[width=0.85\textwidth]{figures/fig_rq3_temporal.pdf}
\caption{Temporal cross-validation results showing consistent APFD performance
across different validation strategies. The narrow range (0.619--0.663) indicates
temporal robustness and minimal concept drift impact. Error bars represent
95\% bootstrap confidence intervals.}
\label{fig:temporal_cv}
\end{figure}

%------------------------------------------------------------------------------
% Figure 5: Hyperparameter Sensitivity (RQ4)
%------------------------------------------------------------------------------
\begin{figure*}[htbp]
\centering
\includegraphics[width=\textwidth]{figures/fig_rq4_sensitivity.pdf}
\caption{Hyperparameter sensitivity analysis. (a) Loss function has the highest
impact on performance ($\Delta$ = 5.9\%). (b) Lower learning rate (3e-5)
outperforms higher rate. (c) Simpler GNN architecture performs better.
(d) 10 selected features optimal. (e) Balanced sampling degrades performance.
(f) Summary of optimal configuration.}
\label{fig:sensitivity}
\end{figure*}

%------------------------------------------------------------------------------
% Figure 6: Qualitative Analysis
%------------------------------------------------------------------------------
\begin{figure}[htbp]
\centering
\includegraphics[width=0.9\textwidth]{figures/fig_qualitative.pdf}
\caption{Qualitative analysis of Filo-Priori performance. (a) APFD score
distribution across 277 builds. (b) First failure detection position distribution.
(c) Failure detection curve showing that running 25\% of tests detects 33.2\%
of failures. (d) Relationship between build size and APFD score.}
\label{fig:qualitative}
\end{figure}

 in your main document
%%%%%%%%%%%%%%%%%%%%%%%%%%%%%%%%%%%%%%%%%%%%%%%%%%%%%%%%%%%%%%%%%%%%%%%%%%%%%%%

%------------------------------------------------------------------------------
% Figure 1: APFD Comparison (RQ1)
%------------------------------------------------------------------------------
\begin{figure}[htbp]
\centering
\includegraphics[width=0.9\textwidth]{figures/fig_rq1_apfd_comparison.pdf}
\caption{Box plot comparison of APFD scores across TCP methods on QTA dataset
(N=277 builds). Filo-Priori achieves competitive performance with traditional
baselines while significantly outperforming recency-based approaches
($p < 0.001$, Wilcoxon signed-rank test). Whiskers extend to 1.5 IQR.}
\label{fig:apfd_comparison}
\end{figure}

%------------------------------------------------------------------------------
% Figure 2: Improvement over Random (RQ1)
%------------------------------------------------------------------------------
\begin{figure}[htbp]
\centering
\includegraphics[width=0.85\textwidth]{figures/fig_rq1_improvement.pdf}
\caption{Relative improvement in APFD over random ordering for each TCP method.
Filo-Priori achieves +10.3\% improvement, significantly outperforming Random
($p < 0.001$). Error bars represent 95\% bootstrap confidence intervals.}
\label{fig:improvement}
\end{figure}

%------------------------------------------------------------------------------
% Figure 3: Ablation Study (RQ2)
%------------------------------------------------------------------------------
\begin{figure}[htbp]
\centering
\includegraphics[width=0.9\textwidth]{figures/fig_rq2_ablation.pdf}
\caption{Component contributions to Filo-Priori performance. Graph Attention
(GATv2) provides the largest contribution (+17.0\%), followed by the Structural
Stream (+5.3\%) and Class Weighting (+4.6\%). Red bars indicate statistically
significant contributions ($p < 0.05$, paired t-test).}
\label{fig:ablation}
\end{figure}

%------------------------------------------------------------------------------
% Figure 4: Temporal Cross-Validation (RQ3)
%------------------------------------------------------------------------------
\begin{figure}[htbp]
\centering
\includegraphics[width=0.85\textwidth]{figures/fig_rq3_temporal.pdf}
\caption{Temporal cross-validation results showing consistent APFD performance
across different validation strategies. The narrow range (0.619--0.663) indicates
temporal robustness and minimal concept drift impact. Error bars represent
95\% bootstrap confidence intervals.}
\label{fig:temporal_cv}
\end{figure}

%------------------------------------------------------------------------------
% Figure 5: Hyperparameter Sensitivity (RQ4)
%------------------------------------------------------------------------------
\begin{figure*}[htbp]
\centering
\includegraphics[width=\textwidth]{figures/fig_rq4_sensitivity.pdf}
\caption{Hyperparameter sensitivity analysis. (a) Loss function has the highest
impact on performance ($\Delta$ = 5.9\%). (b) Lower learning rate (3e-5)
outperforms higher rate. (c) Simpler GNN architecture performs better.
(d) 10 selected features optimal. (e) Balanced sampling degrades performance.
(f) Summary of optimal configuration.}
\label{fig:sensitivity}
\end{figure*}

%------------------------------------------------------------------------------
% Figure 6: Qualitative Analysis
%------------------------------------------------------------------------------
\begin{figure}[htbp]
\centering
\includegraphics[width=0.9\textwidth]{figures/fig_qualitative.pdf}
\caption{Qualitative analysis of Filo-Priori performance. (a) APFD score
distribution across 277 builds. (b) First failure detection position distribution.
(c) Failure detection curve showing that running 25\% of tests detects 33.2\%
of failures. (d) Relationship between build size and APFD score.}
\label{fig:qualitative}
\end{figure}

