
%%%%%%%%%%%%%%%%%%%%%%%%%%%%%%%%%%%%%%%%%%%%%%%%%%%%%%%%%%%%%%%%%%%%%%%%%%%%%%%
% SECTION: RESULTS
%%%%%%%%%%%%%%%%%%%%%%%%%%%%%%%%%%%%%%%%%%%%%%%%%%%%%%%%%%%%%%%%%%%%%%%%%%%%%%%

\section{Results}
\label{sec:results}

This section presents the experimental results organized by research questions.
All experiments were conducted on the QTA dataset containing 277 builds with
at least one failing test case, totaling 8,847 test case executions.

%------------------------------------------------------------------------------
% RQ1: Effectiveness
%------------------------------------------------------------------------------

\subsection{RQ1: How effective is Filo-Priori compared to baseline methods?}
\label{sec:rq1}

Table~\ref{tab:tcp_comparison} presents the comparison of Filo-Priori against
eight baseline methods. We evaluate effectiveness using the Average Percentage
of Faults Detected (APFD) metric, with 95\% bootstrap confidence intervals
and Wilcoxon signed-rank tests for statistical significance.


\textbf{Key Findings:}
\begin{itemize}
    \item Filo-Priori achieves a mean APFD of \textbf{0.6379} (95\% CI: [0.609, 0.669])
    \item This represents a \textbf{14.0\%} improvement over Random ordering (APFD = 0.5596)
    \item The improvement is statistically significant ($p < 0.001$, Wilcoxon signed-rank test)
    \item Filo-Priori outperforms the strongest baseline FailureRate (0.6289) by \textbf{+1.4\%}
    \item Filo-Priori significantly outperforms Recency-based approaches ($p < 0.001$)
\end{itemize}

\begin{table}[htbp]
\centering
\small
\caption{Comparison of TCP Methods on QTA Dataset}
\label{tab:tcp_comparison}
\begin{tabular}{lccccc}
\toprule
Method & Mean APFD & p-value & Cohen's d & Effect & 95\% CI \\
\midrule
\textbf{Filo-Priori} & \textbf{0.6413} & - & - & - & [0.612, 0.672] \\
FailureRate & 0.6289 & 0.3142 & 0.058 & negligible & [0.597, 0.660] \\
XGBoost & 0.6171 & 0.0891 & 0.104 & negligible & [0.583, 0.649] \\
GreedyHistorical & 0.6138 & 0.0412 & 0.126 & negligible & [0.582, 0.647] \\
LogisticRegression & 0.5964 & 0.0087 & 0.186 & small & [0.559, 0.631] \\
RandomForest & 0.5910 & 0.0034 & 0.212 & small & [0.556, 0.627] \\
Random & 0.5596 & 8.42e-05 & 0.342 & small & [0.533, 0.584] \\
RecentFailureRate & 0.5454 & 1.27e-05 & 0.398 & small & [0.515, 0.578] \\
Recency & 0.5240 & 2.18e-09 & 0.502 & medium & [0.490, 0.557] \\
\bottomrule
\end{tabular}
\end{table}



%------------------------------------------------------------------------------
% RQ2: Component Contributions
%------------------------------------------------------------------------------

\subsection{RQ2: What is the contribution of each architectural component?}
\label{sec:rq2}

To understand the importance of each component in Filo-Priori's architecture,
we conducted an ablation study. Table~\ref{tab:ablation} shows the impact
of removing each component on the APFD metric.


\textbf{Key Findings:}
\begin{itemize}
    \item The full model achieves APFD = 0.6397
    \item \textbf{Graph Attention (GATv2)} is the most critical component, contributing +17.0\% to performance
    \item The Structural Stream contributes +5.3\%
    \item Class Weighting contributes +4.6\%
    \item Cross-Attention shows negative contribution (-1.1\%), suggesting simpler fusion may suffice
\end{itemize}

\input{tables/tab_ablation}


%------------------------------------------------------------------------------
% RQ3: Temporal Robustness
%------------------------------------------------------------------------------

\subsection{RQ3: How robust is Filo-Priori across different time periods?}
\label{sec:rq3}

Software projects evolve over time, and a TCP model trained on historical data
must generalize to future builds. We evaluated temporal robustness using three
validation strategies: Temporal K-Fold CV, Sliding Window CV, and Concept Drift
Analysis.


\textbf{Key Findings:}
\begin{itemize}
    \item \textbf{Temporal 5-Fold CV}: APFD = 0.6629 (95\% CI: [0.627, 0.698])
    \item \textbf{Sliding Window CV}: APFD = 0.6279 (95\% CI: [0.595, 0.661])
    \item \textbf{Concept Drift Test}: APFD = 0.6187 (95\% CI: [0.574, 0.661])
    \item Performance remains stable across all temporal validation methods (range: 0.619-0.663)
    \item No significant performance degradation over time, indicating robustness to concept drift
\end{itemize}

\input{tables/tab_temporal_cv}


%------------------------------------------------------------------------------
% RQ4: Hyperparameter Sensitivity
%------------------------------------------------------------------------------

\subsection{RQ4: How sensitive is Filo-Priori to hyperparameter choices?}
\label{sec:rq4}

We analyzed the sensitivity of Filo-Priori to key hyperparameters by comparing
results across multiple experimental configurations.


\textbf{Key Findings:}
\begin{itemize}
    \item \textbf{Loss Function}: Weighted Cross-Entropy performs best (APFD = 0.6191), with sensitivity range $\Delta$ = 0.036
    \item \textbf{Learning Rate}: Lower rate (3e-5) outperforms higher rate (5e-5), $\Delta$ = 0.027
    \item \textbf{GNN Architecture}: Simpler architecture (1 layer, 2 heads) performs best, $\Delta$ = 0.027
    \item \textbf{Structural Features}: 10 selected features outperform both 6 (baseline) and 29 (expanded)
    \item \textbf{Balanced Sampling}: Not recommended for ranking tasks; degrades performance
\end{itemize}

The model shows moderate sensitivity to hyperparameters, with loss function
choice having the largest impact (5.9\% relative variation).

% Sensitivity Analysis Tables - Filo-Priori v9
% Auto-generated


\begin{table}[htbp]
\centering
\caption{Hyperparameter Sensitivity Analysis for Filo-Priori v9}
\label{tab:sensitivity}
\begin{threeparttable}
\begin{tabular}{llccc}
\toprule
\textbf{Hyperparameter} & \textbf{Value} & \textbf{Mean APFD} & \textbf{95\% CI} & \textbf{N} \\
\midrule
Loss Function & Weighted CE & 0.6191 & [0.597, 0.639] & 554 \\
Loss Function & Focal Loss & 0.6117 & [0.587, 0.634] & 554 \\
Loss Function & Weighted Focal & 0.5834 & [0.550, 0.615] & 277 \\
\midrule
Learning Rate & 3e-5 & 0.6160 & [0.599, 0.633] & 831 \\
Learning Rate & 5e-5 & 0.5890 & [0.570, 0.608] & 831 \\
\midrule
GNN Architecture & 1 layer, 2 heads & 0.6160 & [0.598, 0.633] & 831 \\
GNN Architecture & 2 layers, 4 heads & 0.5890 & [0.569, 0.608] & 831 \\
\midrule
Structural Features & 10 features & 0.6171 & [0.589, 0.646] & 277 \\
Structural Features & 6 features & 0.6155 & [0.594, 0.636] & 554 \\
Structural Features & 29 features & 0.5997 & [0.571, 0.630] & 277 \\
\midrule
Balanced Sampling & No Balanced Sampling & 0.6154 & [0.601, 0.631] & 1108 \\
Balanced Sampling & Balanced Sampling & 0.5834 & [0.550, 0.616] & 277 \\
\bottomrule
\end{tabular}
\begin{tablenotes}
\small
\item Note: Best configuration for each hyperparameter shown in bold (highest Mean APFD).
\item N represents the total number of build-level APFD evaluations across relevant experiments.
\end{tablenotes}
\end{threeparttable}
\end{table}


