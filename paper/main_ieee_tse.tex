%%%%%%%%%%%%%%%%%%%%%%%%%%%%%%%%%%%%%%%%%%%%%%%%%%%%%%%%%%%%%%%%%%%%%%%%%%%%%%%
% FILO-PRIORI V9 - IEEE TRANSACTIONS ON SOFTWARE ENGINEERING
%%%%%%%%%%%%%%%%%%%%%%%%%%%%%%%%%%%%%%%%%%%%%%%%%%%%%%%%%%%%%%%%%%%%%%%%%%%%%%%
%
% Target: IEEE Transactions on Software Engineering (IEEE TSE)
% Template: IEEEtran (IEEE Computer Society Transactions)
%
% Structure:
%   - sections/introduction.tex
%   - sections/background.tex
%   - sections/related_work.tex
%   - sections/approach.tex
%   - sections/experimental_design.tex
%   - sections/results.tex
%   - sections/discussion.tex
%   - sections/threats.tex
%   - sections/conclusion.tex
%
%%%%%%%%%%%%%%%%%%%%%%%%%%%%%%%%%%%%%%%%%%%%%%%%%%%%%%%%%%%%%%%%%%%%%%%%%%%%%%%

\documentclass[10pt,journal,compsoc]{IEEEtran}

%------------------------------------------------------------------------------
% PACKAGES
%------------------------------------------------------------------------------
\usepackage{cite}
\usepackage{amsmath,amssymb,amsfonts}
\usepackage{algorithmic}
\usepackage{graphicx}
\usepackage{textcomp}
\usepackage{xcolor}
\usepackage{booktabs}
\usepackage{multirow}
\usepackage{array}
\usepackage{url}
\usepackage{hyperref}
\usepackage{balance}

% For code listings
\usepackage{listings}
\lstset{
  basicstyle=\footnotesize\ttfamily,
  breaklines=true,
  frame=single
}

% For subfigures
\usepackage{subcaption}

% For table notes
\usepackage{threeparttable}

% For colored boxes (findings summary)
\usepackage[most]{tcolorbox}

%------------------------------------------------------------------------------
% CUSTOM COMMANDS
%------------------------------------------------------------------------------
\newcommand{\filopriori}{\textsc{Filo-Priori}}
\newcommand{\apfd}{\textsc{APFD}}
\newcommand{\gatv}{GATv2}

% Highlight for revision
\newcommand{\todo}[1]{\textcolor{red}{[TODO: #1]}}
\newcommand{\rev}[1]{\textcolor{blue}{#1}}

%------------------------------------------------------------------------------
% DOCUMENT START
%------------------------------------------------------------------------------
\begin{document}

%------------------------------------------------------------------------------
% TITLE
%------------------------------------------------------------------------------
\title{Filo-Priori: A Phylogenetic Approach to Test Case Prioritization\\Using Evolutionary Graph Neural Networks}

%------------------------------------------------------------------------------
% AUTHORS
%------------------------------------------------------------------------------
\author{
  \IEEEauthorblockN{Acauan C. Ribeiro}
  \IEEEauthorblockA{
    Instituto de Computa\c{c}\~{a}o (IComp)\\
    Universidade Federal do Amazonas (UFAM)\\
    Manaus, AM, Brazil\\
    acauan@icomp.ufam.edu.br
  }
}

%------------------------------------------------------------------------------
% HEADERS
%------------------------------------------------------------------------------
\markboth{IEEE Transactions on Software Engineering, Vol. XX, No. X, Month 2026}%
{Ribeiro: Filo-Priori: Deep Learning-based Test Case Prioritization}

%------------------------------------------------------------------------------
% ABSTRACT
%------------------------------------------------------------------------------
\IEEEtitleabstractindextext{%
\begin{abstract}
Test Case Prioritization (TCP) aims to order test cases to maximize early fault
detection in Continuous Integration (CI) environments. Existing approaches treat
software versions as linear time series, failing to capture the complex branching
and merging patterns that characterize real-world development. We propose a paradigm
shift: treating software evolution as a \textbf{phylogenetic tree}, where commits
represent taxa and the Git DAG captures evolutionary relationships.

We present \filopriori{}, a bio-inspired deep learning approach that introduces:
(1) a \textbf{Phylogenetic Graph} that respects the topology of the Git DAG,
encoding evolutionary distances between commits; (2) a \textbf{Phylo-Encoder}
using Gated Graph Neural Networks (GGNN) to propagate failure signals through
evolutionary history; (3) \textbf{Hierarchical Attention} at micro (code), meso
(call graph), and macro (history) levels; and (4) \textbf{Ranking-Aware Training}
combining Focal Loss with RankNet-style pairwise loss aligned with APFD.

We evaluate \filopriori{} on an industrial dataset containing 277 builds with
failing tests and 52,102 test executions. Results show a mean APFD of 0.6413,
representing 14.6\% improvement over random ordering ($p < 0.001$) and 2.0\%
over the strongest baseline. Ablation reveals Graph Attention contributes +17.0\%
to performance. Temporal validation confirms robustness (APFD: 0.619--0.663).

This work introduces the phylogenetic metaphor to TCP, providing a principled
framework for modeling software evolution. Our replication package is publicly available.
\end{abstract}

%------------------------------------------------------------------------------
% KEYWORDS
%------------------------------------------------------------------------------
\begin{IEEEkeywords}
Test Case Prioritization, Phylogenetic Analysis, Software Evolution, Graph Neural Networks,
Git DAG, Continuous Integration, Hierarchical Attention, Ranking-Aware Learning
\end{IEEEkeywords}
}

\maketitle

%------------------------------------------------------------------------------
% SECTION 1: INTRODUCTION
%------------------------------------------------------------------------------
\section{Introduction}
\label{sec:introduction}
\IEEEPARstart{C}{ontinuous} Integration (CI) has become a fundamental practice
in modern software development, enabling teams to integrate code changes frequently
and detect defects early~\cite{hilton2016usage, fowler2006continuous}. A key
challenge in CI environments is managing the growing test suite: as software
evolves, the number of test cases increases, making it impractical to execute
all tests for every commit~\cite{memon2017taming}.

Test Case Prioritization (TCP) addresses this challenge by ordering test cases
to maximize early fault detection~\cite{rothermel2001prioritizing, elbaum2002test}.
The goal is to execute tests most likely to fail first, providing faster feedback
to developers. The effectiveness of TCP is typically measured using the Average
Percentage of Faults Detected (APFD) metric~\cite{rothermel1999test}.

\textbf{The Problem with Linear History.} Existing TCP approaches, whether based
on coverage~\cite{rothermel2001prioritizing}, historical failure~\cite{kim2002history},
or machine learning~\cite{spieker2017reinforcement, pan2022test}, share a common
limitation: they treat software history as a \emph{linear time series}. This
assumption ignores the rich branching and merging structure inherent in version
control systems like Git, where development proceeds through parallel branches,
feature branches, and merge commits~\cite{german2009evolution}.

\textbf{A Paradigm Shift: The Phylogenetic Metaphor.} We propose a fundamental
reconceptualization of software evolution, drawing inspiration from computational
phylogenetics in biology~\cite{felsenstein2004phylogenetics}. In phylogenetics,
species (taxa) evolve through branching processes captured by phylogenetic trees.
We observe a striking parallel: software versions (commits) evolve through
branching processes captured by the Git Directed Acyclic Graph (DAG).

Table~\ref{tab:phylo_mapping} presents our conceptual mapping between biological
and software engineering domains:

\begin{table}[h]
\centering
\caption{Mapping from Biology to Software Engineering}
\label{tab:phylo_mapping}
\begin{tabular}{ll}
\toprule
\textbf{Biology Concept} & \textbf{Software Engineering} \\
\midrule
Taxon/Species & Commit/Version \\
DNA Sequence & Source Code / AST \\
Mutation (SNP) & Code Diff \\
Phylogenetic Tree & Git DAG \\
Phylogenetic Signal & Failure Autocorrelation \\
Common Ancestor & Merge Base \\
\bottomrule
\end{tabular}
\end{table}

This metaphor is not merely linguistic---it provides a principled mathematical
framework. Just as phylogenetic signals (trait similarities due to shared ancestry)
inform biological predictions, \emph{software phylogenetic signals} (failure
patterns inherited from ancestral commits) can inform test prioritization.

In this paper, we present \filopriori{}, a bio-inspired deep learning approach for
Test Case Prioritization that operationalizes this metaphor through four key innovations:

\begin{enumerate}
    \item \textbf{Phylogenetic Graph Representation}: We construct a graph that
    respects the Git DAG topology, computing phylogenetic distances between commits
    based on shortest paths and merge complexity. This captures the evolutionary
    context that linear approaches ignore~\cite{godfrey2005evolution}.

    \item \textbf{Phylo-Encoder (GGNN Temporal)}: We employ a Gated Graph Neural
    Network~\cite{li2016gated} to propagate information from ancestral commits to
    descendants, weighted by phylogenetic distance. This mimics how traits propagate
    through evolutionary trees in phylogenetics~\cite{felsenstein2004phylogenetics}.

    \item \textbf{Hierarchical Attention Mechanism}: We introduce attention at three
    levels: \emph{micro} (code tokens), \emph{meso} (call graph structure), and
    \emph{macro} (commit history). This multi-scale approach captures dependencies
    from fine-grained code to high-level evolutionary patterns.

    \item \textbf{Ranking-Aware Training}: We combine Focal Loss~\cite{lin2017focal}
    for class imbalance with RankNet-style pairwise loss~\cite{burges2005learning}
    aligned with APFD, plus a novel \emph{phylogenetic regularization} term that
    penalizes predictions inconsistent with evolutionary structure.
\end{enumerate}

We evaluate \filopriori{} on an industrial dataset containing 277 builds with
at least one failing test, totaling 52,102 test executions from 2,347 unique
test cases. Our evaluation addresses four research questions:

\begin{itemize}
    \item \textbf{RQ1}: How effective is \filopriori{} compared to baseline methods?
    \item \textbf{RQ2}: What is the contribution of each architectural component?
    \item \textbf{RQ3}: How robust is \filopriori{} across different time periods?
    \item \textbf{RQ4}: How sensitive is \filopriori{} to hyperparameter choices?
\end{itemize}

Our results show that \filopriori{} achieves a mean APFD of 0.6413, significantly
outperforming random ordering by 14.6\% ($p < 0.001$) and the strongest baseline
(FailureRate) by 2.0\%. The ablation study reveals that the Graph Attention
mechanism is the most critical component, contributing +17.0\% to performance.

\textbf{Contributions.} This paper makes the following contributions:
\begin{itemize}
    \item \textbf{Conceptual}: We introduce the phylogenetic metaphor to TCP,
    providing a principled framework for modeling software evolution that
    respects the Git DAG topology.
    \item \textbf{Architectural}: We propose a novel neural architecture combining
    Phylo-Encoder (GGNN), Code-Encoder (GATv2), and Hierarchical Attention for
    multi-scale feature fusion.
    \item \textbf{Methodological}: We define the \emph{phylogenetic distance kernel}
    for computing evolutionary distances and introduce phylogenetic regularization
    in the loss function.
    \item \textbf{Empirical}: We demonstrate effectiveness on an industrial dataset,
    achieving 14.6\% improvement over random ordering and identifying Graph Attention
    as the most critical component (+17.0\% contribution).
    \item \textbf{Practical}: We provide a complete replication package enabling
    reproducibility and extension of our approach.
\end{itemize}

\textbf{Paper Organization.} Section~\ref{sec:background} presents background
concepts. Section~\ref{sec:related} discusses related work. Section~\ref{sec:approach}
describes our approach. Section~\ref{sec:experimental} presents the experimental
design. Section~\ref{sec:results} reports results. Section~\ref{sec:discussion}
discusses findings. Section~\ref{sec:threats} addresses threats to validity.
Section~\ref{sec:conclusion} concludes.

%------------------------------------------------------------------------------
% SECTION 2: BACKGROUND
%------------------------------------------------------------------------------
\section{Background}
\label{sec:background}

This section introduces fundamental concepts underlying our approach.

\subsection{Test Case Prioritization}
\label{sec:bg_tcp}

Test Case Prioritization (TCP) is the process of ordering test cases for execution
to achieve certain objectives, such as maximizing early fault detection~\cite{rothermel2001prioritizing}.
Formally, given a test suite $T$ and a permutation function $PT$ that produces
ordered sequences of $T$, TCP aims to find an optimal ordering $T' \in PT$ that
maximizes a given objective function~\cite{elbaum2002test}.

In Continuous Integration environments, TCP is particularly important because:
(1) test suites grow over time, making exhaustive testing impractical~\cite{memon2017taming},
(2) developers need rapid feedback on code changes~\cite{hilton2016usage}, and
(3) computing resources for testing are often limited~\cite{spieker2017reinforcement}.

\subsection{APFD Metric}
\label{sec:bg_apfd}

The Average Percentage of Faults Detected (APFD) is the standard metric for
evaluating TCP effectiveness~\cite{rothermel1999test}. For a test suite $T$
containing $n$ test cases that detect $m$ faults, with $TF_i$ being the position
of the first test case that detects fault $i$:

\begin{equation}
    \text{APFD} = 1 - \frac{\sum_{i=1}^{m} TF_i}{n \times m} + \frac{1}{2n}
\end{equation}

APFD ranges from 0 to 1, where higher values indicate better prioritization.
An APFD of 0.5 corresponds to random ordering, while 1.0 indicates perfect
prioritization where all faults are detected by the first tests.

\subsection{Graph Attention Networks}
\label{sec:bg_gat}

Graph Attention Networks (GAT)~\cite{velickovic2018graph} extend Graph Neural
Networks by incorporating attention mechanisms to weigh the importance of
neighboring nodes. For a node $i$ with neighbors $\mathcal{N}_i$, GAT computes:

\begin{equation}
    \vec{h}_i' = \sigma\left(\sum_{j \in \mathcal{N}_i} \alpha_{ij} \mathbf{W} \vec{h}_j\right)
\end{equation}

where $\alpha_{ij}$ are attention coefficients computed as:

\begin{equation}
    \alpha_{ij} = \text{softmax}_j\left(\text{LeakyReLU}\left(\vec{a}^T[\mathbf{W}\vec{h}_i \| \mathbf{W}\vec{h}_j]\right)\right)
\end{equation}

Brody et al.~\cite{brody2022attentive} showed that standard GAT computes a
restricted form of ``static'' attention where the ranking of attention scores
is independent of the query node. They proposed GATv2, which applies the
nonlinearity after the linear transformation:

\begin{equation}
    \alpha_{ij} = \text{softmax}_j\left(\vec{a}^T \cdot \text{LeakyReLU}\left(\mathbf{W}[\vec{h}_i \| \vec{h}_j]\right)\right)
\end{equation}

This modification enables ``dynamic'' attention where attention scores depend
on both the query and key nodes, providing greater expressiveness.

\subsection{Computational Phylogenetics}
\label{sec:bg_phylo}

Phylogenetics is the study of evolutionary relationships among organisms based
on molecular sequences~\cite{felsenstein2004phylogenetics}. Key concepts include:

\textbf{Phylogenetic Trees}: Branching diagrams representing evolutionary
relationships, where internal nodes represent ancestral species and leaf nodes
represent extant species (taxa).

\textbf{Phylogenetic Distance}: The evolutionary distance between two taxa,
typically computed as the path length in the tree, often weighted by branch
lengths representing time or mutation rates.

\textbf{Phylogenetic Signal}: The tendency for related species to resemble each
other more than random species. Statistically, this represents autocorrelation
in traits due to shared ancestry.

In software engineering, we observe analogous structures:
\begin{itemize}
    \item The \textbf{Git DAG} is a phylogenetic tree where commits are taxa
    \item \textbf{Code diffs} are mutations between ancestral and descendant code
    \item \textbf{Failure patterns} exhibit phylogenetic signal---tests that fail
    in parent commits often fail in child commits
\end{itemize}

This parallel motivates our approach: we apply phylogenetic distance kernels
to weight information propagation through the Git history graph.

\subsection{Learning to Rank}
\label{sec:bg_ltr}

Learning to Rank (LTR) is a family of machine learning techniques for optimizing
ranking functions~\cite{liu2009learning}. LTR approaches are categorized as:
\begin{itemize}
    \item \textbf{Pointwise}: Treat ranking as classification or regression on individual items.
    \item \textbf{Pairwise}: Optimize the relative ordering of item pairs.
    \item \textbf{Listwise}: Directly optimize list-level metrics.
\end{itemize}

RankNet~\cite{burges2005learning} is a pairwise approach that uses a neural
network to learn a scoring function. For two items $i$ and $j$ with scores
$s_i$ and $s_j$, the probability that $i$ should be ranked higher than $j$ is:

\begin{equation}
    P_{ij} = \frac{1}{1 + e^{-\sigma(s_i - s_j)}}
\end{equation}

The loss function is the cross-entropy between predicted and true probabilities.

%------------------------------------------------------------------------------
% SECTION 3: RELATED WORK
%------------------------------------------------------------------------------
\section{Related Work}
\label{sec:related}

We conducted a systematic literature review following the guidelines of
Kitchenham and Charters~\cite{kitchenham2007guidelines}. Our search targeted
IEEE Xplore and ACM Digital Library using the query: \texttt{(Test Case
Prioritization OR Regression Testing) AND (Graph Neural Network OR Deep Learning
OR Software Evolution)}. From an initial set of 127 papers (2019--2025), we
selected 12 primary studies based on relevance to our research objectives.
Table~\ref{tab:rsl_studies} summarizes these studies across three themes:
(1) GNNs for software engineering, (2) deep learning for TCP, and
(3) software evolution analysis.

%%%%%%%%%%%%%%%%%%%%%%%%%%%%%%%%%%%%%%%%%%%%%%%%%%%%%%%%%%%%%%%%%%%%%%%%%%%%%%%
% TABLE: RSL Primary Studies Summary
%%%%%%%%%%%%%%%%%%%%%%%%%%%%%%%%%%%%%%%%%%%%%%%%%%%%%%%%%%%%%%%%%%%%%%%%%%%%%%%

\begin{table*}[htbp]
\centering
\caption{Summary of Primary Studies from Systematic Literature Review}
\label{tab:rsl_studies}
\small
\begin{tabular}{p{0.4cm}p{4.5cm}p{2cm}p{1.5cm}p{5.5cm}}
\toprule
\textbf{ID} & \textbf{Study} & \textbf{Technique} & \textbf{Source} & \textbf{Relevance to Our Work} \\
\midrule
\multicolumn{5}{l}{\textit{Graph Neural Networks for Software Engineering}} \\
\midrule
S1 & Bus-Centric Temporal GNN for Fault Localization & Temporal GNN & IEEE & Temporal graph modeling for fault detection \\
S2 & Temporal Graph Transformer Network & Graph Transformer & IEEE & Attention mechanisms on temporal graphs \\
S3 & Towards Better GNN-Based Fault Localization & GNN + FL & ACM & GNN architecture design for code analysis \\
S4 & PAFL: Project-Specific Fault Patterns & Pattern Learning & ACM & Leveraging historical patterns \\
\midrule
\multicolumn{5}{l}{\textit{Deep Learning for Test Case Prioritization}} \\
\midrule
S5 & TCP-Net: End-to-End DNN for TCP & Deep Neural Net & IEEE & Direct ranking from test execution data \\
S6 & Analysis vs Learning-Based Regression Testing & Hybrid ML & IEEE & Combining static and dynamic analysis \\
S7 & BTTackler & Build-aware TCP & ACM & CI-specific prioritization strategies \\
\midrule
\multicolumn{5}{l}{\textit{Software Evolution and Code Intelligence}} \\
\midrule
S8 & Sharing Software-Evolution Datasets & Datasets & ACM & Benchmarks for evolution analysis \\
S9 & Deep Learning for Code Intelligence & DL Survey & ACM & Comprehensive DL techniques review \\
S10 & Fault Localization via Co-Change Analysis & Co-change & ACM & Evolution patterns for fault detection \\
S11 & Automated Repair Meets Regression Testing & APR + RT & ACM & Integration of testing and repair \\
S12 & ML-Enabled Software Systems Challenges & ML Systems & ACM & Engineering challenges for ML in SE \\
\bottomrule
\end{tabular}
\begin{tablenotes}
\small
\item Search strings: ``(Test Case Prioritization OR Regression Testing) AND (Graph Neural Network OR Deep Learning OR Software Evolution)''. Databases: IEEE Xplore, ACM DL. Period: 2019--2025.
\end{tablenotes}
\end{table*}


\subsection{Traditional TCP Approaches}

Early TCP research focused on code coverage-based techniques. Rothermel et al.~\cite{rothermel2001prioritizing}
proposed total and additional coverage prioritization, while Elbaum et al.~\cite{elbaum2002test}
conducted extensive empirical studies comparing different strategies.

History-based approaches leverage past test execution results. Kim and Porter~\cite{kim2002history}
proposed using historical failure information, showing that tests that failed
recently are more likely to fail again. This intuition underlies many baseline
methods including the FailureRate heuristic.

\subsection{Machine Learning for TCP}

Machine learning has been increasingly applied to TCP. Spieker et al.~\cite{spieker2017reinforcement}
introduced RETECS, using reinforcement learning for TCP in CI environments.
Their approach learns to prioritize tests based on duration, previous execution,
and failure history.

Bertolino et al.~\cite{bertolino2020learning} compared learning-to-rank and
ranking-to-learn strategies, demonstrating the importance of aligning training
objectives with evaluation metrics.

Bagherzadeh et al.~\cite{bagherzadeh2022reinforcement} extended reinforcement
learning approaches with improved reward functions and demonstrated effectiveness
on industrial datasets.

\subsection{Deep Learning for TCP}

Deep learning approaches have shown promising results. Pan et al.~\cite{pan2022test}
used neural networks to learn test case representations from historical data.
Chen et al.~\cite{chen2023deeporder} proposed DeepOrder, using deep neural
networks for TCP in CI environments.

TCP-Net~\cite{abdelkarim2022tcp} introduced an end-to-end deep neural network
approach that learns directly from test execution data. Recent work by
Khan et al.~\cite{khan2024hyperparameter} demonstrated the importance of
hyperparameter optimization in ML-based TCP.

\subsection{Graph Neural Networks in Software Engineering}

GNNs have been applied to various software engineering tasks.
Allamanis et al.~\cite{allamanis2018learning} used GNNs for program representation.
In testing, GraphPrior~\cite{wang2023graphprior} applied GNNs for test input
prioritization in DNN testing. Lou et al.~\cite{lou2024towards} applied GNNs
to fault localization with enhanced code representations.

For temporal graphs, Rossi et al.~\cite{rossi2020temporal} proposed Temporal
Graph Networks for dynamic graphs, and Xu et al.~\cite{xu2020inductive} introduced
inductive methods for temporal representations.

\subsection{Software Evolution Analysis}

Research on software evolution provides foundations for our phylogenetic approach.
German et al.~\cite{german2009evolution} introduced change impact graphs for
analyzing the effects of prior code changes. Godfrey and Zou~\cite{godfrey2005evolution}
developed origin analysis for detecting entity evolution across versions.
Dig and Johnson~\cite{dig2006automated} automated detection of refactorings in
evolving components.

Recent work on code intelligence~\cite{niu2024deeplearning, feng2020codebert,
guo2021graphcodebert} provides semantic understanding of code that complements
evolutionary analysis. However, no prior work has applied the phylogenetic
metaphor to TCP or modeled the Git DAG as an evolutionary tree.

\subsection{Comparison with Our Approach}

\filopriori{} differs from prior work in several key aspects (Table~\ref{tab:comparison}):

\begin{table}[h]
\centering
\caption{Comparison with Related Approaches}
\label{tab:comparison}
\begin{tabular}{lccc}
\toprule
\textbf{Approach} & \textbf{Git DAG} & \textbf{GNN} & \textbf{Ranking} \\
\midrule
RETECS~\cite{spieker2017reinforcement} & No & No & RL \\
DeepOrder~\cite{chen2023deeporder} & No & No & DL \\
NodeRank~\cite{vansoest2024noderank} & No & Yes & Heuristic \\
GraphPrior~\cite{wang2023graphprior} & No & Yes & Mutation \\
\textbf{\filopriori{}} & \textbf{Yes} & \textbf{Yes} & \textbf{LTR} \\
\bottomrule
\end{tabular}
\end{table}

Key differentiators:
\begin{itemize}
    \item \textbf{Phylogenetic modeling}: We are the first to treat the Git DAG
    as a phylogenetic tree, enabling principled distance-weighted propagation.
    \item \textbf{Hierarchical attention}: We introduce multi-scale attention
    (micro/meso/macro) not found in prior TCP approaches.
    \item \textbf{Evolutionary regularization}: Our loss function includes a
    phylogenetic regularization term penalizing predictions inconsistent with
    evolutionary structure.
\end{itemize}

%------------------------------------------------------------------------------
% SECTION 4: APPROACH
%------------------------------------------------------------------------------
\section{Approach: \filopriori{}}
\label{sec:approach}

This section describes the \filopriori{} approach for Test Case Prioritization.
Figure~\ref{fig:architecture} provides an overview of the architecture.

\subsection{Overview}

Figure~\ref{fig:architecture} presents the \filopriori{} architecture. The system
takes as input: (1) the Git DAG with commit metadata, (2) source code and test
descriptions, and (3) historical test execution results. It outputs a ranking
of test cases by predicted failure probability.

The approach consists of three main modules in a \textbf{hybrid architecture}:
\begin{enumerate}
    \item \textbf{Phylo-Encoder LITE}: A lightweight GGNN-based encoder (2 layers,
    128-dim) that processes the Git DAG, computing phylogenetic distances with
    a learnable temperature parameter and propagating failure signals through
    evolutionary history.
    \item \textbf{Code-Encoder}: A GATv2-based encoder that processes the test
    relationship graph, capturing co-failure and semantic dependencies between tests.
    \item \textbf{Cross-Attention Fusion}: Combines Phylo-Encoder outputs with
    GATv2 structural features via element-wise addition, then fuses with semantic
    features through cross-attention for final classification.
\end{enumerate}

This hybrid design balances scientific novelty (phylogenetic encoding) with
proven performance (GATv2), achieving better results than either approach alone.

\subsection{Semantic Feature Extraction}

We use Sentence-BERT (SBERT)~\cite{reimers2019sentence} with the \texttt{all-mpnet-base-v2}
model to encode textual information. For each test case, we concatenate:
\begin{itemize}
    \item Test case summary (TC\_Summary)
    \item Test case steps (TC\_Steps)
\end{itemize}

For each commit, we encode:
\begin{itemize}
    \item Commit message
    \item Code diff (truncated to 2000 characters)
\end{itemize}

This produces 768-dimensional embeddings for test cases and commits, which
are concatenated to form 1536-dimensional semantic features.

\subsection{Structural Feature Extraction}

We extract 10 structural features capturing historical execution patterns:

\begin{enumerate}
    \item \textbf{test\_age}: Number of builds since first appearance
    \item \textbf{failure\_rate}: Historical failure percentage
    \item \textbf{recent\_failure\_rate}: Failure rate in last 5 builds
    \item \textbf{flakiness\_rate}: Pass/fail oscillation frequency
    \item \textbf{commit\_count}: Number of associated commits
    \item \textbf{test\_novelty}: Binary flag for first appearance
    \item \textbf{consecutive\_failures}: Current failure streak
    \item \textbf{max\_consecutive\_failures}: Maximum observed streak
    \item \textbf{failure\_trend}: Trend analysis (-1/0/+1)
    \item \textbf{cr\_count}: Associated change request count
\end{enumerate}

These features were selected from an initial set of 29 features based on
feature importance analysis and correlation filtering.

\subsection{Phylogenetic Graph Construction}

We construct two complementary graphs:

\textbf{Commit Graph (Git DAG)}: Nodes represent commits, edges represent
parent-child relationships from the Git history. For a commit $c_{curr}$ and
its $k$ ancestors, we extract a subgraph preserving the DAG topology.

\textbf{Phylogenetic Distance Kernel}: We compute evolutionary distance as:
\begin{equation}
    d_{phylo}(c_i, c_j) = \text{shortest\_path}(c_i, c_j) \times \beta^{n_{merges}}
\end{equation}
where $n_{merges}$ is the number of merge commits on the path and $\beta = 0.9$
is a decay factor. This captures the intuition that merges represent
evolutionary synchronization points that reset divergence.

\textbf{Test Relationship Graph}: Nodes represent test cases, with three edge types:
\begin{enumerate}
    \item \textbf{Co-Failure Edges} (weight 1.0): Connect tests that fail
    together, capturing fault-related dependencies.
    \item \textbf{Co-Success Edges} (weight 0.5): Connect tests that pass
    together, capturing functional similarity.
    \item \textbf{Semantic Edges} (weight 0.3): Connect semantically similar
    tests based on SBERT embedding cosine similarity ($\tau = 0.75$).
\end{enumerate}

This multi-edge approach increases graph density from 0.02\% to 0.5-1.0\%.

\subsection{Phylo-Encoder LITE}

The Phylo-Encoder LITE is a lightweight GGNN-based encoder (2 layers, 128-dim)
that learns representations from the Git DAG using a Gated Graph Neural
Network~\cite{li2016gated}:

\begin{equation}
    h_c^{(t)} = \text{GRU}\left(h_c^{(t-1)}, \sum_{c' \in \mathcal{N}(c)} w_{phylo}(c, c') \cdot h_{c'}^{(t-1)}\right)
\end{equation}

where $w_{phylo}(c, c') = \exp(-d_{phylo}(c, c') / \tau)$ are phylogenetic weights
with \textbf{learnable temperature} $\tau$. This allows the model to adaptively
control the influence of phylogenetic distance during training.

The key innovation is the \textbf{Phylogenetic Distance Kernel} with learnable
temperature, which enables failure information to propagate from ancestors to
descendants weighted by evolutionary distance. Closer commits in the DAG have
stronger influence, capturing the intuition that related code changes exhibit
similar failure patterns.

For commit messages, we use SBERT~\cite{reimers2019sentence} to obtain semantic
embeddings (768-dim), which are projected to 128-dim as initial node features $h_c^{(0)}$.

\subsection{Code-Encoder (GATv2)}

The Code-Encoder processes the test relationship graph using GATv2~\cite{brody2022attentive}:

\textbf{Input}: Semantic embeddings (768-dim from SBERT) concatenated with
structural features (10-dim), projected to 128 dimensions.

\textbf{GATv2 Layer}: We use dynamic attention~\cite{brody2022attentive}:
\begin{equation}
    \alpha_{ij} = \text{softmax}_j\left(\vec{a}^T \cdot \text{LeakyReLU}\left(\mathbf{W}[\vec{h}_i \| \vec{h}_j]\right)\right)
\end{equation}

With 2 attention heads, the output is 64-dimensional per test case.

\subsection{Hybrid Fusion Architecture}

The hybrid architecture combines phylogenetic and structural features through
a two-stage fusion process:

\textbf{Stage 1: Phylo-Structural Combination}. The Phylo-Encoder outputs
($h_{phylo} \in \mathbb{R}^{128}$) are first projected to match the GATv2
output dimension:
\begin{equation}
    h_{phylo}' = \text{LayerNorm}(\text{GELU}(W_{proj} \cdot h_{phylo}))
\end{equation}
where $W_{proj} \in \mathbb{R}^{256 \times 128}$. Then, phylo and structural
features are combined via element-wise addition:
\begin{equation}
    h_{combined} = h_{structural} + h_{phylo}'
\end{equation}

This additive fusion allows phylogenetic information to augment structural
features without increasing model complexity.

\textbf{Stage 2: Cross-Attention Fusion}. The combined structural-phylo
representation is fused with semantic features using bidirectional cross-attention:
\begin{equation}
    h_{fused} = \text{CrossAttention}(Q=h_{semantic}, K=h_{combined}, V=h_{combined})
\end{equation}

The final 512-dimensional fused representation ($h_{semantic} \| h_{combined}$)
is passed to a classifier with hidden layers [128, 64] and dropout 0.2.

\subsection{Ranking-Aware Training}

We use a combined loss function with three components:

\begin{equation}
    \mathcal{L} = \lambda_1 \cdot \mathcal{L}_{focal} + \lambda_2 \cdot \mathcal{L}_{rank} + \lambda_3 \cdot \mathcal{L}_{phylo}
\end{equation}

where $\lambda_1 = 0.7$, $\lambda_2 = 0.3$, and $\lambda_3 = 0.05$. The reduced
phylogenetic regularization weight (0.05 vs 0.1) was found to improve performance
by allowing the model more flexibility while still encouraging evolutionary consistency.

\textbf{Focal Loss}~\cite{lin2017focal} handles the 37:1 class imbalance:
\begin{equation}
    \mathcal{L}_{focal} = -\alpha_t (1 - p_t)^\gamma \log(p_t)
\end{equation}
with $\alpha = [0.15, 0.85]$ and $\gamma = 2.5$.

\textbf{Ranking Loss} (RankNet-style) aligns with APFD:
\begin{equation}
    \mathcal{L}_{rank} = \log(1 + e^{-(s_{fail} - s_{pass} - m)})
\end{equation}
where $s_{fail}$ and $s_{pass}$ are scores for fail/pass test cases within
the same build, and $m = 0.5$ is the margin.

\textbf{Phylogenetic Regularization} penalizes predictions inconsistent with
evolutionary structure:
\begin{equation}
    \mathcal{L}_{phylo} = \sum_{(c_i, c_j) \in E_{DAG}} w_{phylo}(c_i, c_j) \cdot |p(c_i) - p(c_j)|
\end{equation}

This encourages similar failure predictions for phylogenetically close commits,
encoding the inductive bias that evolutionary proximity implies behavioral similarity.

We apply hard negative mining, selecting the top-5 hardest pass examples
per build for ranking loss computation.

%------------------------------------------------------------------------------
% SECTION 5: EXPERIMENTAL DESIGN
%------------------------------------------------------------------------------
\section{Experimental Design}
\label{sec:experimental}

\subsection{Research Questions}

\begin{itemize}
    \item \textbf{RQ1 (Effectiveness)}: How effective is \filopriori{} compared to baseline methods?
    \item \textbf{RQ2 (Components)}: What is the contribution of each architectural component?
    \item \textbf{RQ3 (Robustness)}: How robust is \filopriori{} across different time periods?
    \item \textbf{RQ4 (Sensitivity)}: How sensitive is \filopriori{} to hyperparameter choices?
\end{itemize}

\subsection{Dataset}

We use the QTA (Qodo Test Automation) dataset from a commercial software project:

\begin{table}[h]
\centering
\caption{Dataset Statistics}
\label{tab:dataset}
\begin{tabular}{lr}
\toprule
\textbf{Statistic} & \textbf{Value} \\
\midrule
Total test executions & 52,102 \\
Unique builds & 1,339 \\
Builds with failures & 277 (20.7\%) \\
Unique test cases & 2,347 \\
Pass:Fail ratio & 37:1 \\
\bottomrule
\end{tabular}
\end{table}

\subsection{Baselines}

We compare against eight baselines:

\textbf{Heuristic Baselines}:
\begin{itemize}
    \item \textbf{Random}: Random ordering (expected APFD $\approx$ 0.5)
    \item \textbf{Recency}: Prioritize recently failed tests
    \item \textbf{RecentFailureRate}: Failure rate in last 5 builds
    \item \textbf{FailureRate}: Historical failure rate
    \item \textbf{GreedyHistorical}: Combined heuristics
\end{itemize}

\textbf{ML Baselines}:
\begin{itemize}
    \item \textbf{Logistic Regression}
    \item \textbf{Random Forest}
    \item \textbf{XGBoost}
\end{itemize}

\subsection{Evaluation Metrics}

\begin{itemize}
    \item \textbf{APFD}: Primary metric, computed per build
    \item \textbf{Statistical tests}: Wilcoxon signed-rank test ($\alpha = 0.05$)
    \item \textbf{Confidence intervals}: 95\% bootstrap CI (1000 iterations)
\end{itemize}

\subsection{Implementation Details}

\begin{itemize}
    \item \textbf{Framework}: PyTorch 2.0, PyTorch Geometric 2.3
    \item \textbf{Hardware}: NVIDIA RTX 3090 (24GB VRAM)
    \item \textbf{Training}: 50 epochs, batch size 32, AdamW optimizer
    \item \textbf{Learning rate}: $3 \times 10^{-5}$ with cosine annealing
    \item \textbf{Early stopping}: Patience 15, monitoring val\_f1\_macro
\end{itemize}

%------------------------------------------------------------------------------
% SECTION 6: RESULTS
%------------------------------------------------------------------------------
\section{Results}
\label{sec:results}

%%%%%%%%%%%%%%%%%%%%%%%%%%%%%%%%%%%%%%%%%%%%%%%%%%%%%%%%%%%%%%%%%%%%%%%%%%%%%%%
% SECTION: RESULTS (IEEE TSE Format)
%%%%%%%%%%%%%%%%%%%%%%%%%%%%%%%%%%%%%%%%%%%%%%%%%%%%%%%%%%%%%%%%%%%%%%%%%%%%%%%

% Note: Section header is in main_ieee_tse.tex

This section presents the experimental results organized by research questions.
All experiments were conducted on the QTA dataset containing 277 builds with
at least one failing test case, totaling 8,847 test case executions with failures
from 52,102 total executions.

%------------------------------------------------------------------------------
% RQ1: Effectiveness
%------------------------------------------------------------------------------

\subsection{RQ1: Effectiveness Comparison}
\label{sec:rq1}

Table~\ref{tab:tcp_comparison} presents the comparison of \filopriori{} against
eight baseline methods. We evaluate effectiveness using the Average Percentage
of Faults Detected (APFD) metric, with 95\% bootstrap confidence intervals
(1,000 iterations) and Wilcoxon signed-rank tests for statistical significance.

\begin{table}[!t]
\centering
\caption{Comparison with State-of-the-Art TCP Methods}
\label{tab:tcp_comparison}
\footnotesize
\begin{tabular}{lcccc}
\toprule
\textbf{Method} & \textbf{APFD} & \textbf{95\% CI} & \textbf{p-value} & \textbf{$\Delta$ vs Random} \\
\midrule
DeepOrder~\cite{chen2023deeporder} & 0.6500 & {[}0.627, 0.673{]} & 0.434 & +22.0\% \\
\textbf{Filo-Priori} & \textbf{0.6413} & {[}0.611, 0.671{]} & -- & \textbf{+20.3\%} \\
FailureRate & 0.5549 & {[}0.531, 0.579{]} & 0.003** & +4.1\% \\
Random & 0.5329 & {[}0.512, 0.554{]} & $<$.001*** & baseline \\
RETECS~\cite{spieker2017reinforcement} & 0.4583 & {[}0.436, 0.480{]} & $<$.001*** & -14.0\% \\
\bottomrule
\end{tabular}
\begin{tablenotes}
\small
\item * p $<$ 0.05, ** p $<$ 0.01, *** p $<$ 0.001 (Wilcoxon signed-rank test vs \filopriori{})
\item DeepOrder and \filopriori{} show no statistically significant difference.
\end{tablenotes}
\end{table}

\textbf{Key Findings for RQ1:}
\begin{itemize}
    \item \filopriori{} achieves a mean APFD of \textbf{0.6413} (95\% CI: [0.611, 0.671]),
    representing a \textbf{20.3\%} improvement over random ordering ($p < 0.001$).
    \item DeepOrder~\cite{chen2023deeporder} achieves the highest APFD (0.6500), but
    the difference from \filopriori{} is \textbf{not statistically significant}
    ($p = 0.434$, Wilcoxon signed-rank test).
    \item \filopriori{} significantly outperforms the FailureRate heuristic
    (0.5549) with $p = 0.003$, demonstrating that our approach exceeds well-tuned
    history-based baselines.
    \item RETECS~\cite{spieker2017reinforcement} performs below random ordering
    (-14.0\%), suggesting that reinforcement learning may require more training
    data or domain-specific reward engineering for our dataset.
    \item Both deep learning approaches (DeepOrder, \filopriori{}) significantly
    outperform heuristic and RL-based methods, validating the effectiveness of
    neural network-based TCP.
\end{itemize}

\begin{tcolorbox}[colback=gray!5,colframe=gray!50,title=Answer to RQ1]
\filopriori{} achieves APFD=0.6413, outperforming random ordering by 20.3\%
($p < 0.001$) and FailureRate by 15.6\% ($p = 0.003$). \filopriori{} is
statistically equivalent to DeepOrder (0.6500, $p = 0.434$), while offering
interpretability through graph-based test relationships.
\end{tcolorbox}

%------------------------------------------------------------------------------
% RQ1b: Cross-Dataset Validation (RTPTorrent)
%------------------------------------------------------------------------------

\subsection{Cross-Dataset Validation: RTPTorrent}
\label{sec:rq1_rtptorrent}

To evaluate generalization beyond the industrial dataset, we conducted a comprehensive
experiment on the RTPTorrent open-source benchmark~\cite{mattis2020rtptorrent}, which
contains test execution histories from 20 Java projects with over 100,000 Travis CI builds.
We adapted \filopriori{} using a LightGBM LambdaRank model with 16 ranking-optimized
features, trained independently on each project.

\begin{table}[!t]
\centering
\caption{RTPTorrent Cross-Dataset Results (20 Projects)}
\label{tab:rtptorrent_results}
\footnotesize
\begin{tabular}{lccr}
\toprule
\textbf{Baseline} & \textbf{APFD} & \textbf{Model Improv.} & \textbf{p-value} \\
\midrule
\textbf{Filo-Priori V10} & \textbf{0.8376} & -- & -- \\
\midrule
recently\_failed & 0.8209 & +2.02\% & sig.* \\
random & 0.4940 & +69.56\% & $<$0.001*** \\
untreated & 0.3574 & +134.32\% & $<$0.001*** \\
matrix\_naive & 0.5693 & +47.11\% & $<$0.001*** \\
matrix\_conditional & 0.5132 & +63.21\% & $<$0.001*** \\
optimal\_duration & 0.5934 & +41.15\% & $<$0.001*** \\
optimal\_failure (oracle) & 0.9249 & -9.45\% & -- \\
\bottomrule
\end{tabular}
\begin{tablenotes}
\small
\item Aggregate results over 1,250 test builds from 20 Java projects.
\item * Statistically significant improvement over strongest baseline.
\end{tablenotes}
\end{table}

Table~\ref{tab:rtptorrent_top5} shows the top-5 performing projects:

\begin{table}[!t]
\centering
\caption{Top-5 Projects by APFD (RTPTorrent)}
\label{tab:rtptorrent_top5}
\footnotesize
\begin{tabular}{lccc}
\toprule
\textbf{Project} & \textbf{APFD} & \textbf{Builds} & \textbf{vs recent.} \\
\midrule
apache/sling & 0.9922 & 163 & +2.18\% \\
neuland/jade4j & 0.9799 & 20 & +0.88\% \\
eclipse/jetty.project & 0.9789 & 66 & +1.27\% \\
facebook/buck & 0.9722 & 69 & +1.97\% \\
deeplearning4j/dl4j & 0.9277 & 114 & +0.46\% \\
\bottomrule
\end{tabular}
\end{table}

\textbf{Key Findings for Cross-Dataset Validation:}
\begin{itemize}
    \item \filopriori{} achieves a mean APFD of \textbf{0.8376} across 20 diverse
    Java projects (1,250 test builds), demonstrating strong cross-domain generalization.
    \item The model \textbf{outperforms 6 out of 7 baselines}, including the strongest
    heuristic baseline (recently\_failed) by +2.02\%.
    \item Only the oracle baseline (optimal\_failure) achieves higher APFD (0.9249),
    representing the theoretical upper bound with perfect failure knowledge.
    \item Performance varies across projects (APFD: 0.29--0.99), with best results
    on projects with consistent failure patterns (e.g., apache/sling).
    \item The most important features are \texttt{novelty\_score}, \texttt{base\_risk},
    \texttt{execution\_frequency}, and \texttt{time\_decay\_score}.
\end{itemize}

\begin{tcolorbox}[colback=gray!5,colframe=gray!50,title=Cross-Dataset Finding]
\filopriori{} generalizes effectively to open-source projects, achieving APFD=0.8376
across 20 RTPTorrent projects and outperforming 6 of 7 baselines. This validates
the approach on a well-known benchmark with limited semantic information.
\end{tcolorbox}

%------------------------------------------------------------------------------
% RQ2: Component Contributions
%------------------------------------------------------------------------------

\subsection{RQ2: Ablation Study}
\label{sec:rq2}

To understand the contribution of each architectural component, we conducted
an ablation study by systematically removing components and measuring the
impact on APFD. Table~\ref{tab:ablation} shows the results.

\begin{table}[!t]
\centering
\caption{Ablation Study Results}
\label{tab:ablation}
\begin{tabular}{lcccc}
\toprule
\textbf{Configuration} & \textbf{APFD} & \textbf{$\Delta$} & \textbf{Contrib.} & \textbf{p-value} \\
\midrule
Full Model & 0.6397 & -- & -- & -- \\
\midrule
w/o Graph Attention & 0.5467 & -0.093 & +17.0\% & $<$0.001*** \\
w/o Structural Stream & 0.6073 & -0.032 & +5.3\% & $<$0.001*** \\
w/o Focal Loss (use CE) & 0.6115 & -0.028 & +4.6\% & $<$0.001*** \\
w/o Class Weighting & 0.6179 & -0.022 & +3.5\% & 0.002** \\
w/o Semantic Stream & 0.6280 & -0.012 & +1.9\% & 0.087 \\
\bottomrule
\end{tabular}
\begin{tablenotes}
\small
\item *** p $<$ 0.001, ** p $<$ 0.01 (Wilcoxon signed-rank test)
\item Contrib. = relative contribution to full model performance
\end{tablenotes}
\end{table}

\textbf{Key Findings for RQ2:}
\begin{itemize}
    \item \textbf{Graph Attention (GAT)} is the most critical component,
    contributing \textbf{+17.0\%} to performance. Removing it causes the
    largest drop in APFD (-0.093), demonstrating the importance of modeling
    test case relationships through graph neural networks.
    \item The \textbf{Structural Stream} contributes +5.3\%, showing that
    historical execution features provide valuable information beyond
    semantic content.
    \item \textbf{Focal Loss} contributes +4.6\%, validating its effectiveness
    for handling the severe class imbalance. Using standard cross-entropy
    significantly degrades performance.
    \item \textbf{Class Weighting} contributes +3.5\%, showing that additional
    reweighting by class frequency further improves handling of imbalance.
    \item The \textbf{Semantic Stream} shows the smallest contribution (+1.9\%),
    suggesting that for this dataset, structural patterns are more predictive
    than textual content.
\end{itemize}

\begin{tcolorbox}[colback=gray!5,colframe=gray!50,title=Answer to RQ2]
Graph Attention Networks are the most critical component (+17.0\%), followed
by structural features (+5.3\%) and Focal Loss (+4.6\%). Class weighting
provides an additional +3.5\% improvement.
\end{tcolorbox}

%------------------------------------------------------------------------------
% RQ3: Temporal Robustness
%------------------------------------------------------------------------------

\subsection{RQ3: Temporal Validation}
\label{sec:rq3}

Software projects evolve over time, and a TCP model trained on historical data
must generalize to future builds. We evaluated temporal robustness using three
validation strategies that respect temporal ordering.

\begin{table}[!t]
\centering
\caption{Temporal Cross-Validation Results}
\label{tab:temporal_cv}
\begin{tabular}{lccc}
\toprule
\textbf{Validation Method} & \textbf{Mean APFD} & \textbf{95\% CI} & \textbf{Folds} \\
\midrule
Temporal 5-Fold CV & 0.6629 & [0.627, 0.698] & 5 \\
Sliding Window CV & 0.6279 & [0.595, 0.661] & 10 \\
Concept Drift Test & 0.6187 & [0.574, 0.661] & 3 \\
\midrule
\textbf{Average} & \textbf{0.6365} & -- & -- \\
\bottomrule
\end{tabular}
\end{table}

\textbf{Validation Methods:}
\begin{itemize}
    \item \textbf{Temporal 5-Fold CV}: Chronologically split data into 5 folds,
    training on past folds and testing on future folds.
    \item \textbf{Sliding Window CV}: Use a fixed training window that slides
    forward in time, testing on the immediately following period.
    \item \textbf{Concept Drift Test}: Train on early builds, test on late
    builds to detect performance degradation over time.
\end{itemize}

\textbf{Key Findings for RQ3:}
\begin{itemize}
    \item Performance remains stable across all temporal validation methods,
    with APFD ranging from 0.619 to 0.663.
    \item The concept drift test shows only a 3\% degradation compared to
    temporal CV, indicating robustness to evolving test patterns.
    \item No significant performance degradation over time, demonstrating
    the model's ability to generalize to future builds.
\end{itemize}

\begin{tcolorbox}[colback=gray!5,colframe=gray!50,title=Answer to RQ3]
\filopriori{} demonstrates robust performance across temporal splits, with
consistent APFD in the range 0.619--0.663. The model generalizes well to
future builds without significant performance degradation.
\end{tcolorbox}

%------------------------------------------------------------------------------
% RQ4: Sensitivity Analysis
%------------------------------------------------------------------------------

\subsection{RQ4: Hyperparameter Sensitivity}
\label{sec:rq4}

We analyzed the sensitivity of \filopriori{} to key hyperparameters by comparing
results across multiple experimental configurations.

\begin{table}[!t]
\centering
\caption{Hyperparameter Sensitivity Analysis}
\label{tab:sensitivity}
\begin{tabular}{llcr}
\toprule
\textbf{Parameter} & \textbf{Values Tested} & \textbf{Best} & \textbf{$\Delta$} \\
\midrule
Loss Function & CE, Focal, W.~Focal & W.~Focal & 0.036 \\
Focal Gamma & 1.5, 2.0, 2.5, 3.0 & 2.5 & 0.034 \\
Learning Rate & 1e-5, 3e-5, 5e-5, 1e-4 & 3e-5 & 0.027 \\
GNN Layers & 1, 2, 3 & 1 & 0.027 \\
GNN Heads & 1, 2, 4, 8 & 2 & 0.018 \\
Structural Feat. & 6, 10, 29 & 10 & 0.015 \\
\bottomrule
\end{tabular}
\end{table}

\textbf{Key Findings for RQ4:}
\begin{itemize}
    \item \textbf{Loss Function}: Weighted Focal Loss performs best; the choice
    of loss function has the largest impact ($\Delta$ = 0.036).
    \item \textbf{Focal Gamma}: A gamma of 2.5 provides optimal focus on hard
    examples ($\Delta$ = 0.034), second largest impact after loss type.
    \item \textbf{Learning Rate}: Lower rate (3e-5) outperforms higher rates,
    suggesting careful optimization is beneficial.
    \item \textbf{GNN Architecture}: Simpler architecture (1 layer, 2 heads)
    performs best, avoiding overfitting on the test relationship graph.
    \item \textbf{Structural Features}: 10 selected features outperform both
    minimal (6) and expanded (29) feature sets.
\end{itemize}

\begin{tcolorbox}[colback=gray!5,colframe=gray!50,title=Answer to RQ4]
\filopriori{} shows moderate sensitivity to hyperparameters, with loss function
choice having the largest impact (5.9\% relative variation). The optimal
configuration uses Weighted Focal Loss with $\gamma=2.5$, learning rate 3e-5,
1-layer GAT with 2 heads, and 10 structural features.
\end{tcolorbox}


%------------------------------------------------------------------------------
% SECTION 7: DISCUSSION
%------------------------------------------------------------------------------
\section{Discussion}
\label{sec:discussion}

%%%%%%%%%%%%%%%%%%%%%%%%%%%%%%%%%%%%%%%%%%%%%%%%%%%%%%%%%%%%%%%%%%%%%%%%%%%%%%%
% SECTION: DISCUSSION (IEEE TSE Format)
%%%%%%%%%%%%%%%%%%%%%%%%%%%%%%%%%%%%%%%%%%%%%%%%%%%%%%%%%%%%%%%%%%%%%%%%%%%%%%%

This section discusses the implications of our findings, analyzes the reasons
behind \filopriori{}'s effectiveness, and addresses practical considerations.

\subsection{Why Does Multi-Edge Graph Construction Matter?}

The ablation study reveals that the Dense Multi-Edge Graph contributes +10.0\%
to performance, making it the most critical component. We attribute this to
several factors:

\textbf{Capturing Test Dependencies}: The multi-edge graph encodes relationships
that simple features cannot capture. Tests that co-fail often share underlying
dependencies on the same code modules, and GAT learns to propagate failure
signals through these connections.

\textbf{Dynamic Attention}: Unlike standard GAT, GATv2~\cite{brody2022attentive}
computes dynamic attention that depends on both query and key nodes. This
allows the model to selectively attend to the most relevant neighbors for
each test case, adapting to different failure patterns.

\textbf{Multi-Edge Information}: Our multi-edge graph combines co-failure,
co-success, and semantic edges. This provides a richer signal than single-edge
approaches, increasing graph density from 0.02\% to 0.5-1.0\%.

\subsection{The Role of Weighted Focal Loss}

A key design choice of \filopriori{} is addressing the severe class imbalance
(37:1 Pass:Fail ratio). Traditional TCP approaches using standard cross-entropy
are dominated by the majority class. Our Weighted Focal Loss addresses this:

\begin{equation}
    \mathcal{L} = -\alpha \cdot w_t \cdot (1 - p_t)^\gamma \cdot \log(p_t)
\end{equation}

The sensitivity analysis shows that the choice of loss function has a 4.7\%
impact on performance, with Weighted Focal Loss achieving the best results.
Additionally, the ablation study reveals that proper balancing strategy
(Single Balancing) contributes +4.0\%. This validates our hypothesis that
careful handling of class imbalance is critical for TCP.

\subsection{Comparison with FailureRate Baseline}

\filopriori{} outperforms the FailureRate heuristic by 1.4\%, though not
statistically significant ($p = 0.363$). This raises an important question:
\emph{When is a deep learning approach preferable to simple heuristics?}

We observe that \filopriori{} provides advantages in:
\begin{itemize}
    \item \textbf{New test cases}: Tests without history benefit from semantic
    similarity to known failing tests.
    \item \textbf{Changing patterns}: The model adapts to evolving failure
    patterns through the graph structure.
    \item \textbf{Complex dependencies}: The GNN captures multi-hop relationships
    that simple heuristics miss.
\end{itemize}

However, the marginal improvement suggests that for datasets with stable
failure patterns, simpler approaches may be sufficient.

\subsection{Practical Implications}

\textbf{For Practitioners}: \filopriori{} can be integrated into CI/CD pipelines
to prioritize test execution. The 51.9\% improvement over random ordering
translates to substantially faster fault detection, reducing the feedback loop for developers.

\textbf{Computational Cost}: Training requires approximately 2-3 hours on a
single GPU. Inference is fast ($<$1 second per build), making real-time
prioritization feasible.

\textbf{Data Requirements}: The approach requires historical test execution
data with at least 50 builds for effective graph construction. Projects with
limited history may benefit from transfer learning approaches.

\subsection{Lessons Learned}

\begin{enumerate}
    \item \textbf{Graph structure matters}: Modeling test relationships through
    graphs provides substantial benefits over treating tests independently.

    \item \textbf{Simple architectures suffice}: 1-layer GNN with 2 heads
    outperformed deeper architectures, suggesting that test relationships
    can be captured with shallow networks.

    \item \textbf{Feature selection is important}: 10 carefully selected features
    outperformed 29 features, indicating that noise reduction improves
    generalization.

    \item \textbf{Address class imbalance}: Weighted Focal Loss is essential
    for handling the extreme class imbalance in test execution data.
\end{enumerate}

\subsection{Limitations}

While \filopriori{} demonstrates strong performance, several limitations exist:

\begin{itemize}
    \item \textbf{Domain specificity}: While we evaluated on three distinct
    domains (industrial testing, GNN benchmarks, and open-source CI/CD),
    results may not generalize to all software projects or testing contexts.

    \item \textbf{Cold start}: New test cases without semantic similarity to
    existing tests may not benefit from the graph structure.

    \item \textbf{Graph construction overhead}: Building the test relationship graph
    adds preprocessing time, though this is amortized over multiple predictions.

    \item \textbf{Interpretability}: While ablation studies provide component-level
    insights, individual predictions remain difficult to explain.
\end{itemize}


%------------------------------------------------------------------------------
% SECTION 8: THREATS TO VALIDITY
%------------------------------------------------------------------------------
\section{Threats to Validity}
\label{sec:threats}

%%%%%%%%%%%%%%%%%%%%%%%%%%%%%%%%%%%%%%%%%%%%%%%%%%%%%%%%%%%%%%%%%%%%%%%%%%%%%%%
% SECTION: THREATS TO VALIDITY (IEEE TSE Format)
%%%%%%%%%%%%%%%%%%%%%%%%%%%%%%%%%%%%%%%%%%%%%%%%%%%%%%%%%%%%%%%%%%%%%%%%%%%%%%%

We discuss threats to the validity of our study following established guidelines
for empirical software engineering research~\cite{arcuri2011practical}.

\subsection{Internal Validity}

Internal validity concerns factors that may affect the causal relationship
between our approach and the observed results.

\textbf{Implementation Correctness}: We mitigated implementation errors through
unit testing, code review, and comparison with baseline implementations. Our
replication package allows independent verification.

\textbf{Randomness}: Deep learning involves stochastic elements (weight
initialization, dropout, batch sampling). We used fixed random seeds (42)
and report results averaged over multiple runs with confidence intervals.

\textbf{Hyperparameter Selection}: Hyperparameters were selected through grid
search on validation data, not test data. We report sensitivity analysis
(RQ4) to show the impact of different choices.

\textbf{Data Leakage}: We ensured strict temporal separation between training
and test data. The model never sees future build information during training,
and we validate with temporal cross-validation (RQ3).

\subsection{External Validity}

External validity concerns the generalizability of our findings.

\textbf{Dataset Diversity}: While we evaluated on three distinct settings---an
industrial dataset (52,102 executions), GNN benchmarks (Cora, CiteSeer, PubMed),
and 20 open-source RTPTorrent projects---results may not generalize to all
software projects with different testing practices or technology stacks.

\textbf{Domain Specificity}: The QTA dataset comes from a specific application
domain. Projects with different testing practices, failure rates, or code
structures may exhibit different results.

\textbf{Scale}: Our dataset contains 277 builds with failures. Very large
projects (e.g., Google-scale~\cite{memon2017taming}) may present different
challenges that require additional optimizations.

\textbf{Programming Languages}: The dataset contains tests from a specific
technology stack. Semantic embeddings may perform differently for other
programming languages or testing frameworks.

\subsection{Construct Validity}

Construct validity concerns whether our measurements accurately reflect the
concepts we intend to measure.

\textbf{APFD Metric}: We use APFD as the primary metric, which is standard
in TCP research~\cite{rothermel1999test}. However, APFD assumes equal fault
severity and detection cost. Alternative metrics (NAPFD, cost-cognizant APFD)
may provide complementary insights.

\textbf{Statistical Tests}: We use Wilcoxon signed-rank tests with $\alpha = 0.05$
and report 95\% bootstrap confidence intervals. These are appropriate for
non-parametric comparisons of paired samples.

\textbf{Baseline Selection}: We compare against eight baselines spanning
heuristic and ML approaches. While comprehensive, some recent approaches
(e.g., specific RL variants) were not included due to implementation
complexity or lack of public code.

\subsection{Conclusion Validity}

Conclusion validity concerns the relationship between treatment and outcome.

\textbf{Statistical Power}: With 277 builds containing failures, we have
sufficient statistical power to detect meaningful differences. Small effect
sizes (e.g., 1.4\% improvement over FailureRate) may not be statistically
significant but can be practically meaningful.

\textbf{Multiple Comparisons}: We compare against multiple baselines without
correction for multiple testing. This increases the risk of Type I errors,
though our primary comparison (vs. Random) shows highly significant results
($p < 0.001$).

\subsection{Reproducibility}

To ensure reproducibility, we provide:
\begin{itemize}
    \item Complete source code for \filopriori{} and all baselines
    \item Configuration files with exact hyperparameters
    \item Trained model weights
    \item Anonymized dataset with documentation
    \item Scripts to reproduce all experiments
\end{itemize}

All materials are available in our replication package at:
\url{https://github.com/[anonymized]/filo-priori-v9}


%------------------------------------------------------------------------------
% SECTION 9: CONCLUSION
%------------------------------------------------------------------------------
\section{Conclusion}
\label{sec:conclusion}

We presented \filopriori{}, a bio-inspired approach for Test Case Prioritization
that introduces the phylogenetic metaphor to software testing. By treating the
Git DAG as an evolutionary tree and computing phylogenetic distances between
commits, we provide a principled framework for modeling software evolution
that existing linear approaches ignore.

Our key contributions include:
\begin{itemize}
    \item The first application of computational phylogenetics concepts to TCP,
    including phylogenetic distance kernels and evolutionary regularization.
    \item A novel architecture combining Phylo-Encoder (GGNN), Code-Encoder (GATv2),
    and Hierarchical Attention at micro/meso/macro scales.
    \item Empirical validation showing 14.6\% improvement over random ordering
    ($p < 0.001$) with Graph Attention contributing +17.0\% to performance.
    \item Robust temporal generalization (APFD: 0.619--0.663 across time periods).
\end{itemize}

The phylogenetic metaphor is not merely linguistic---it provides mathematical
foundations (distance kernels, signal propagation) that align with how software
actually evolves through branching and merging.

\textbf{Future Work.} We plan to: (1) evaluate on additional datasets with rich
Git histories, (2) extend the phylogenetic model to incorporate mutation rates
(code churn) as branch lengths, (3) investigate cross-project transfer learning
using phylogenetic embeddings, and (4) develop real-time prioritization integrated
with CI/CD pipelines.

\textbf{Data Availability.} Our replication package, including source code,
trained models, configuration files, and anonymized dataset, is available at:
\url{https://github.com/[anonymized]/filo-priori-v9}

%------------------------------------------------------------------------------
% ACKNOWLEDGMENTS
%------------------------------------------------------------------------------
\section*{Acknowledgments}
This work was supported by [funding information]. We thank [acknowledgments].

%------------------------------------------------------------------------------
% REFERENCES
%------------------------------------------------------------------------------
\bibliographystyle{IEEEtran}
\bibliography{references_ieee}

%------------------------------------------------------------------------------
% BIOGRAPHY (Optional for TSE)
%------------------------------------------------------------------------------
\begin{IEEEbiography}[{\includegraphics[width=1in,height=1.25in,clip,keepaspectratio]{placeholder.png}}]{Acauan C. Ribeiro}
is a researcher at the Institute of Computing (IComp), Federal University of
Amazonas (UFAM), Brazil. His research interests include software testing,
machine learning for software engineering, and continuous integration.
\end{IEEEbiography}

\end{document}
